%!TEX root = ../Thesis.tex
\section{Anforderungen und Abgrenzung}
Die Anforderungen dieses Projekts gehen auf konkrete Bedürfnisse einer Nutzergruppe zurück, 
die im Arbeitsalltag intensiv mit dem System „Newport“ arbeitet. Aus diesen Bedürfnissen ergibt sich das zentrale Ziel dieser Arbeit.

\subsection{Zielsetzung}
Im Rahmen dieser Bachelor-Arbeit soll das Minimum Viable Product (MVP) des Projekts erarbeitet werden, also 
eine erste lauffähige Version mit den Kernfunktionen, die von der Nutzergruppe als unverzichtbar herausgearbeitet 
wurden. Entsprechend liegt in der Anforderungskonzeption der Fokus auf den sogenannten „Must-Have“-Anforderungen,
ohne welche die Anwendung ihren Zweck verfehlen würde. 

Die durch interne Gespräche erfassten Nutzeranforderungen lassen sich in zwei Kategorien einteilen. An erster Stelle stehen
die funktionalen Anforderungen, die sich direkt auf die Interaktion mit der Anwendung beziehen. Ebenfalls wichtig sind jedoch auch
nicht-funktionale Anforderungen, die sich auf Qualität, Sicherheit, Performance und Wartbarkeit der Lösung beziehen.

\subsection{Funktionale Anforderungen}
Im Rahmen des Minimum Viable Products (MVP) wurden folgende funktionale Anforderungen als unverzichtbar identifiziert. 
Sie bilden die Grundlage für die Nutzung des Systems durch Fachanwendende ohne tiefgehendes technisches Vorwissen. 
\begin{itemize}
  \item \textbf{Anlegen neuer dynamischer Attribute:}  
Nutzer*innen sollen die Möglichkeit haben, eigene Attribute anzulegen, die einem bestehenden Projektelement 
(z.\,B. einer Formulierung oder einem Projekt) zugewiesen werden können. Hierbei müssen folgende Angaben möglich sein:\footcite{Nadkarni2001}

  \begin{table}[h!]
  \centering
  \begin{tabular}{|l|c|}
  \hline
  \textbf{Angabe}         & \textbf{Pflichtfeld} \\
  \hline
  Name des Attributs      & Ja               \\
  Beschreibung            & Nein                 \\
  Datentyp (z.\,B. Text, Zahl, Datum) & Ja    \\
  \hline
  \end{tabular}
  \caption{Erforderliche Angaben beim Anlegen eines neuen Attributs}
  \end{table}

  \item \textbf{Zuweisung von Attributen zu Datenobjekten:}  
Bereits definierte Attribute sollen spezifischen Datenbankeinträgen zugeordnet und durch nutzerspezifische Werte ergänzt werden. Dies umfasst:
\begin{itemize}
  \item Auswahl eines vorhandenen Attributs
  \item Eingabe eines entsprechenden Werts
  \item Validierung des Werts entsprechend des Datentyps
\end{itemize}\footcite[Vgl.][]{LiBox2005}

  \item \textbf{Bearbeiten und Anzeigen von Attribut-Werten:}  
Im Benutzerinterface sollen alle bereits zugewiesenen Attribute mitsamt ihren Werten für ein bestimmtes Projektelement angezeigt und editierbar sein. 
Die Darstellung soll dynamisch erfolgen und sich an der Anzahl und Art der zugewiesenen Attribute orientieren.\footcite[Vgl.][]{LiBox2005}

  \item \textbf{Löschen von Attribut-Zuweisungen:} 
Nutzer*innen sollen die Möglichkeit haben, bestehende Attribut-Wert-Paare wieder zu entfernen, ohne dabei das globale Attribut selbst zu löschen. 
Dadurch bleibt das Attribut für andere Einträge verfügbar.\footcite[Vgl.][]{ErdemFinin2009}

  \item \textbf{Benutzerführung und Validierung:} 
Die Oberfläche muss durch verständliche Beschriftungen, Platzhaltertexte, Tooltips oder andere visuelle Hinweise den Nutzer bei der Eingabe unterstützen. 
Validierungsfehler sollen unmittelbar und verständlich angezeigt werden.\footcite[Vgl.][]{Nielsen2014}
\end{itemize}

\subsection{Nicht-funktionale Anforderungen}
Neben den funktionalen Anforderungen, die die direkten Systemfähigkeiten betreffen, wurden verschiedene nicht-funktionale Anforderungen identifiziert. Diese beschreiben die Qualitätsmerkmale des Systems und sind essenziell 
für den erfolgreichen Betrieb in einem unternehmenskritischen Umfeld wie dem von Bayer.

\begin{itemize}
  \item \textbf{Performance:}  
  Das System muss eine hohe Reaktionsgeschwindigkeit aufweisen, um eine flüssige Nutzererfahrung zu gewährleisten.  
  Die Ladezeit für die Anzeige eines Projektelements inklusive dynamischer Felder soll unter 500 ms liegen (bei bis zu 20 zusätzlichen Attributen).  
  Schreiboperationen (z.\,B. Hinzufügen eines Attribut-Werts) sollen ohne spürbare Verzögerung erfolgen.\footcite[Vgl.][]{Galletta2006}

  \item \textbf{Skalierbarkeit:}  
  Die Lösung soll so ausgelegt sein, dass sie mit wachsender Datenmenge und Nutzerzahl ohne grundlegende Umstrukturierung betrieben werden kann.  
  Die Datenbankstruktur muss große Mengen an dynamischen Attribut-Werten performant verarbeiten können.  
  Die Architektur (Frontend, Backend, Datenbank) muss eine horizontale Skalierung ermöglichen (z.\,B. API-Lastverteilung).\footcite[Vgl.][]{Leavitt2010}

  \item \textbf{Datenintegrität:}  
  Es muss sichergestellt werden, dass sämtliche gespeicherte Daten vollständig, korrekt und konsistent sind.  
  Alle Eingaben durch die Nutzer:innen müssen auf Plausibilität und Formatkonformität geprüft werden.  
  Die Datenbank muss durch Constraints und Foreign Keys inkonsistente Zustände verhindern.\footcite[Vgl.][]{Date2003}

  \item \textbf{Wartbarkeit:}  
  Die Codebasis soll modular und verständlich strukturiert sein, sodass zukünftige Weiterentwicklungen oder Fehlerbehebungen effizient durchgeführt werden können.  
  Der Quellcode muss dokumentiert sein (Kommentare, Readmes, API-Spezifikationen).  
  Wiederverwendbare Komponenten und Services sollen sauber gekapselt sein.\footcite[Vgl.][]{Parnas1972}

  \item \textbf{Sicherheit:}  
  Das System muss grundlegende Sicherheitsanforderungen erfüllen, um unberechtigte Zugriffe zu verhindern und sensible Daten zu schützen.  
  Schreibende Aktionen (z.\,B. das Anlegen neuer Attribute) sollen nur autorisierten Nutzern erlaubt sein.  
  Die API muss gegen typische Angriffe (z.\,B. SQL-Injection, CSRF) abgesichert sein.\footcite[Vgl.][]{OWASP2017}

  \item \textbf{Benutzerfreundlichkeit (Usability):}  
  Die Benutzeroberfläche muss so gestaltet sein, dass auch nicht-technische Anwender:innen effektiv mit dem System arbeiten können.  
  Komplexe Prozesse wie das Anlegen neuer Felder sollen durch schrittweise geführte Dialoge unterstützt werden.  
  Fehlerzustände (z.\,B. ungültige Eingaben) müssen klar und nachvollziehbar kommuniziert werden.\footcite[Vgl.][]{Nielsen2016}
\end{itemize}

\subsection{Abgrenzung}
Es ist zwingend notwendig, sich im Rahmen dieser Bachelorarbeit auf das MVP der Anwendung zu fokussieren, um eine abschließende
Evaluation zu gewährleisten. Dementsprechend gibt es relevante Anforderungen an das fertige Produkt, die im Rahmen dieser Arbeit 
nicht umgesetzt werden. All diese Anforderungen sollen im Folgenden der Vollständigkeit halber aufgeführt und begründet werden.

\begin{itemize}
  \item \textbf{Rollen- und Berechtigungssystem:} 
  Viele verschiedene Nutzer werden für ihre jeweiligen Vorhaben in dieser Anwendung Attribute anlegen und diesen Werte zuweisen. Da künftig auch vertrauliche Daten gespeichert 
  werden könnten, ist perspektivisch ein Berechtigungssystem erforderlich. Vorerst werden Nutzer jedoch angewiesen, keine solchen Daten zu speichern, bis dieses System fertig 
  implementiert ist, was voraussichtlich erst nach Ende des Bearbeitungszeitraums dieser Arbeit geschehen wird.
  
  \item \textbf{Internationalisierung:} 
  Das fertige Produkt soll, wie auch in vielen anderen Anwendungen der Fall (z.\,B. SAP), in vielen verschiedenen Sprachen verfügbar sein. 
  Das soll erreicht werden, indem dynamische Language Files eingesetzt werden, die auch eigens von Nutzern bereitgestellt werden können. Eine solche Implementierung 
  ist jedoch zeitaufwendig und aufgrund der Sprachkenntnisse der Nutzer nicht von elementarer Wichtigkeit und wird deshalb in dieser Arbeit keine Verwendung finden.
  
  \item \textbf{Integration in andere Systeme:} 
  Das Projekt soll zwar publiziert werden, um Nutzer-Rückmeldungen erhalten zu können, jedoch soll es noch nicht in die restliche IT-Landschaft integriert werden,
  da das über lange Zeit und unter großer Vorsicht geschehen muss, um Kompatibilität zu garantieren.
\end{itemize}

Zusätzlich zu den hier aufgeführten Anforderungen wurden Erfolgskriterien definiert, anhand derer die Zielerreichung in Kapitel 4 überprüft wird.
