%!TEX root = ../Thesis.tex
\section{Einleitung}
Relationale Datenbanken sind weiterhin die Standardlösung für die Speicherung und Verarbeitung betrieblicher Daten in Unternehmen \footcite{Prasad2025}.  
Durch ihre rigide Struktur ermöglichen sie eine klare Trennung zwischen Datenstruktur und Anwendungsebene, wodurch eine hohe Datenintegrität sichergestellt wird \footcite{Codd1970}; \footcite{Date2003}.  
Diese rigide Struktur kann jedoch auch zu Einschränkungen führen, insbesondere wenn Tabellen zur Laufzeit erweitert werden müssten, um eine flexible und nutzerspezifische Erweiterung zu ermöglichen \footcite{Nadkarni2001}.

Auch im betrieblichen Ablauf der Bayer AG sind relationale Datenbanken mitsamt ihrer Einschränkungen allgegenwärtig.  
Dies gilt insbesondere für Abteilungen, die für Research and Development (R\&D) zuständig sind.  
Alle Projekte, die im Rahmen von Forschungsaktivitäten durchgeführt werden, werden mitsamt zusätzlicher Attribute in einer Datenbank (Newport) hinterlegt, wodurch Budgetierung und Projekt-Planung effizient und effektiv möglich ist.  
Zwar besteht die Möglichkeit, dort viele Informationen zu hinterlegen, jedoch lassen sich die vorgegebenen Spalten nicht erweitern.  
Dadurch ist es nicht möglich, benutzerspezifische Informationen, die über den Horizont von Newport hinausgehen, zu speichern und weiterzuverarbeiten.  
Um dieses Problem zu lösen, soll eine Benutzeroberfläche entwickelt werden, die diese Speicherung ermöglicht.

Im Rahmen dieser Bachelor-Arbeit soll ein flexibles technisches Framework konzipiert und umgesetzt werden, mit welchem benutzerspezifische Attribut-Erweiterungen ermöglicht werden, ohne die bestehende Datenbankstruktur der Newport-Datenbank zu verändern.  
Dabei soll untersucht werden, wie sich die eingeführte Flexibilität mit den Anforderungen an Datenintegrität und Benutzerfreundlichkeit vereinbaren lassen.

Die Arbeit gliedert sich in folgende Kapitel:  
Zuerst sollen in Kapitel 2 wichtige grundlegende Konzepte und Begriffe rund um relationale Datenbanken und flexible Datenmodellierung erläutert werden.  
Daraufhin werden die Anforderungen an das zu entwickelnde Framework sowie die Zielsetzung für das Projekt definiert.  
Die folgenden Kapitel widmen sich der technischen Konzeption und Implementierung des Front- und Backends sowie einer abschließenden Evaluation der Ergebnisse.  
Abschließend folgt ein Ausblick auf mögliche zukünftige Entwicklungen.
