%!TEX root = ../Thesis.tex
\section{Einleitung}
Datenspeicherung und Verarbeitung ist ein zentraler Bestandteil des betrieblichen Ablaufs der Bayer AG.
Dies gilt insbesondere für Abteilungen, die für Research and Development (R\&D) zuständig sind. Alle Projekte,
die im Sinne der Forschung durchgeführt werden, werden mitsamt zusätzlicher Attribute in einer Datenbank (Newport) 
hinterlegt, wodurch Budgetierung und Projekt-Planung effizient und effektiv möglich ist. Zwar besteht die Möglichkeit,
viele Informationen in Newport zu hinterlegen, jedoch lassen sich die vorgegebenen Spalten nicht erweitern. Dadurch ist es 
nicht möglich, benutzerspezifische Informationen, die über den Horizont von Newport hinausgehen, zu speichern und verwenden.
Um dieses Problem zu lösen, soll eine Benutzeroberfläche entwickelt werden, die diese Speicherung ermöglicht. Im Folgenden 
soll da Projekt vollumfänglich von Konzeption bis Umsetzung beschrieben, und Abläufe und Prinzipien erklärt werden.
