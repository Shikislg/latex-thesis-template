%!TEX root = ../Thesis.tex
\section{Fazit}
Im Rahmen dieser Bachelor-Arbeit wurde, basierend auf den definierten Anforderungen und Zielen, ein flexibles Framework aufgebaut, welches in Zukunft 
eine Vielzahl verschiedener Anwendungen auf Basis des unterstützten Datenstamms ermöglicht. Für die speziell angeforderte Anwendung konnte ein MVP erstellt werden, 
der in den kommenden Monaten verbessert und erweitert werden kann. Die Implementierung der dynamischen Attribute ermöglicht es,
dass Nutzer*innen ihre Projektelemente flexibel anpassen können, ohne dass dafür eine neue Datenbankstruktur entworfen werden muss. In Zukunft wird das Framework 
ermöglichen, das Bayer-Portfolio um viele weitere Funktionalitäten zu erweitern, mit denen die Produktivität der Division erheblich gesteigert werden kann. Insgesamt
leistet das Ergebnis der Arbeit einen praktischen Beitrag zur Selbstständigkeit von Fachabteilungen bei der Datenverarbeitung und legt das Fundament für zukünftige 
Fortschritte in der Softwareentwicklung bei Bayer.