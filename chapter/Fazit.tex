%!TEX root = ../Thesis.tex
\section{Fazit}
Im Rahmen dieser Bachelor-Arbeit sollte ein technisches Framework entwickelt werden, mit welchem es möglich ist, benutzerspezifische Informationen 
dynamisch zu bestehenden Datensätzen zu speichern, ohne die zugrunde liegende Datenstruktur anpassen zu müssen.

Im Zentrum stand dabei die Frage, inwieweit diese Flexibilität einen Einschnitt in die Integrität der Daten, die Benutzerfreundlichkeit und Wartbarkeit 
des Systems darstellt. Die Ergebnisse der Arbeit zeigen, dass eine solche Lösung technisch realisierbar und im praktischen Betrieb stabil einsetzbar ist.

Sowohl das Datenmodell als auch das Frontend wurden so konzipiert, dass Erweiterungen möglich sind, ohne dabei bestehende Strukturen verändern zu müssen.
Der modulare Aufbau der Anwendung ermöglicht es, neue Komponenten mit überschaubarem Aufwand zu integrieren.
Besonderer Fokus lag auf einer klaren Trennung zwischen API-Kommunikation, Benutzeroberfläche und der Geschäftslogik - wodurch die Wartbarkeit 
langfristig gesichert wird. 

In der Evaluation konnte gezeigt werden, dass alle funktionalen Kernanforderungen erfüllt wurden. 
Auch unter Last und nicht-idealen Bedingungen (z. B. langsamer Verbindung oder inkorrekter Nutzereingaben) zeigt sich die Anwendung robus und nutzbar.

Trotzdem bleiben weiterhin offene Punkte: Die Durchsatzrate der API ist noch nicht ausreichend, um die prognostizierte maximale Last vollständig zu bewältigen.
Darüber hiknaus ist langfristig ein Rollen- und Rechtesystem erforderlich, um den Schutz sensibler Daten zu gewährleisten.

Insgesamt zeigt diese Arbeit, dass dynamische Erweiterbarkeit und technische Robustheit kein WIderspruch sein müssen, solange Struktur, Validierung und Nutzerführung
in der Architektur von Beginn an berücksichtigt werden.