%!TEX root = ../Thesis.tex
\section{Datenmodellierung}
Im Sinne der Modularität basiert die Datenmodellierung auf dem Entity-Value-Attribut-Konzept (EAV). Dieses Konzept ermöglicht die Erweiterung 
des starren relationalen Schema um eine variable Anzahl benutzerdefinierter Attribute, ohne dabei eine Schema-Veränderung zu erfordern. Dadurch 
sind die dynamischen Attribute von den Domänenobjekten (Projekte und Formulationen) getrennt. Im Folgenden soll der Aufbau und die Funktionsweise
des Datenmodells genauer erläutert werden.
\subsection{ER-Modell und Tabellenübersicht}
Das relationale Schema gliedert sich in 3 Hauptblöcke: Domänentabellen, Attributdefinition und der Werteebene. 
\subsubsection{Domänentabellen}
\enquote{Newport\_Project} verbindet die Datenbank über den Primärschlüssel \enquote{ProjectID} mit der Newport-Datenbank aus dem CropScience Warehouse (CSW). 
Das ermöglicht indirekten Zugriff auf externe Projektdaten. \enquote{Newport\_Segments} beinhaltet sogenannte Segmente\footnote{Segmente sind in der Enwticklung einer Formulation beispielsweise verschiedene Länder, in denen das Produkt lizensiert werden soll} 
für die jeweiligen Newport-Projekte, und (sofern gegeben) deren Attribute. \enquote{Formulation} und \enquote{Formulation\_Segments} fungieren entsprechend für 
zu speichernde Formulationen.
\subsubsection{Attributdefinition}
\enquote{Attribute} enthält alle potentiell verwendbaren, dynamischen Felder. Jede Zeile verfügt über eine eindeutige \enquote{AttributeID}, einen entsprechenden \enquote{AttributeTitle},
und sowohl eine direkte (\enquote{ReferenceType}), als auch eine indirekte Referenz(\enquote{ReferenceID}) auf den Attribut-Owner. Mit der \enquote{SpecID} wird zusätzlich
auf eine Tabelle verwiesen, in der genauere Informationen über das Attribut (bspw. Einheit) hinterlegt werden können.
\subsubsection{Werteebene}
\enquote{Value} speichert die benutzerdefinierten Werte, mit denen das Attribut angelegt wurde. Wesentliche Spalten sind \enquote{ValueID}, der Primärschlüssel, \enquote{AttributeID},
die Referenz auf das zugehörige Attribut, \enquote{Description} für eine mögliche genauere Beschreibung und \enquote{Value}, worin der eigentliche Wert gespeichert wird.
\subsection{Beziehungen und Referentialität}
Ein Domänenobjekt kann über die AttributeID mit beliebig vielen Attributen verbunden werden. Ein Attribut kann widerum nur einem Domänenobjekt zugewiesen werden. 
Die referentielle Integrität wird durch die Fremdschlüssel-Paarungen sichergestellt, wonach kein Attribut ohne Verweis auf ein Domänenobjekt erstellt werden kann. 
Die API ist hingegen dafür zuständig, dass ein Attribut nicht auf mehrere verschiedene Domänenobjekte verweist, indem es benutzerdefiniertes Setzen der AttributeID verhindert.
\subsection{Datenkonsistenz und Validierung}
Das EAV-Modell sieht vor, dass die Datenintegrität durch die mit der Datenbank verbundene API sichergestellt wird, indem eingegebene Werte stets auf Validität geprüft werden.
Folgende Prüfungen sind für die API vorgesehen:
\subsubsection{Datentypprüfung}
Jedes Attribut hat in der Specification einen hinterlegten Datentyp. Sollte der vom Benutzer angegebene Datentyp nicht in der Menge unterstützter Datentypen vorhanden sein,
soll die Erstellung des Attributs verhindert werden und eine entsprechende Fehlermeldung ausgegeben.
\subsubsection{Pflichtfelder und Standardwerte}
Um eine reibungslose Behandlung fehlerhafter Nutzerangaben sicherzustellen, soll schon das Front-End Pflichtfelder als solche markieren, und das Abschicken der Form verhindern,
bis diese ausgefüllt sind. Das Setzen der Schlüssel übernimmt jedoch das Backend, weshalb diese, obwohl sie Pflichtfelder sind, nicht vom Nutzer gesetzt werden müssen.
\subsubsection{Transaktionen}
Das Anlegen eines neuen Attributs und das direkte Zuweisen eines ersten Werts sollen in einer Transaktion gebündelt sein, um zu verhindern, dass Attribute ohen zugehörige Werte 
existieren.
\subsection{Beispielhafte Speicherung}
Angenommen, ein Nutzer legt für \enquote{Projekt A} ein neues Attribut \enquote{Versuchsdauer} (Einheit Integer) an und vergibt den Wert \enquote{3},
dann sähen die Abläufe und Datenbankeinträge folgendermaßen aus:
\begin{table}[h]
    \centering
    \caption{Attributdefinition}
    \label{tab:attribute}
    \begin{tabular}{@{}cccc@{}}
        \toprule
        \textbf{AttributeID} & \textbf{AttributeTitle} & \textbf{ReferenceType} & \textbf{SpecID} \\ 
        \midrule
        42 & Versuchsdauer & Newport\_Project & 2 \\ 
        \bottomrule
    \end{tabular}
\end{table}
\begin{table}[h]
    \centering
    \caption{Specification (Einheitenzuordnung)}
    \label{tab:specification}
    \begin{tabular}{@{}cc@{}}
        \toprule
        \textbf{SpecID} & \textbf{Unit} \\ 
        \midrule
        2 & Integer \\ 
        \bottomrule
    \end{tabular}
\end{table}
\begin{table}[h]
    \centering
    \caption{Speicherung eines Attribut-Wert-Paares}
    \label{tab:value}
    \begin{tabular}{@{}ccccc@{}}
        \toprule
        \textbf{ValueID} & \textbf{AttributeID} & \textbf{Description} & \textbf{Value} \\ 
        \midrule
        101 & 42 & null & 3 \\ 
        \bottomrule
    \end{tabular}
\end{table}

Durch diese Trennung bleiben das feste Schema in \enquote{Newport\_Project} und die dynamischen Erweiterungen 
klar voneinander getrennt, und trotzdem werden alle zusätzlichen relevanten Informationen abgespeichert.

Mit diesem Datenmodell können beliebig viele neue Attribute hinzugefügt werden, ohne das Grundschema der Domänentabelle anzupassen.
